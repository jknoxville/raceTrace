%Useful latex commands from demo diss:
%\noindent followed by a paragraph wont indent the start of it
%{\tt put text here} will put it in monospace(?) font
%\begin{samepage} presumably makes an atomic block
%see demo diss for making latex compile simple block diagrams etc. protocol state machines?
%also see it for adding diagrams. recommends using postscript

\documentclass[12pt,twoside,notitlepage]{report}

\usepackage{a4}
\usepackage{verbatim}
\usepackage{url}

\input{epsf}                            % to allow postscript inclusions
% On thor and CUS read top of file:
%     /opt/TeX/lib/texmf/tex/dvips/epsf.sty
% On CL machines read:
%     /usr/lib/tex/macros/dvips/epsf.tex



\raggedbottom                           % try to avoid widows and orphans
\sloppy
\clubpenalty1000%
\widowpenalty1000%

\addtolength{\oddsidemargin}{6mm}       % adjust margins
\addtolength{\evensidemargin}{-8mm}

\renewcommand{\baselinestretch}{1.1}    % adjust line spacing to make
                                        % more readable

\begin{document}

\bibliographystyle{plain}


%%%%%%%%%%%%%%%%%%%%%%%%%%%%%%%%%%%%%%%%%%%%%%%%%%%%%%%%%%%%%%%%%%%%%%%%
% Title


\pagestyle{empty}

\hfill{\LARGE \bf John Knox}

\vspace*{60mm}
\begin{center}
\Huge
{\bf RaceTrace:\\A real-time group tracking app} \\
\vspace*{5mm}
Computer Science Tripos, Part II \\
\vspace*{5mm}
Churchill College \\
\vspace*{5mm}
\today  % today's date
\end{center}

\cleardoublepage

%%%%%%%%%%%%%%%%%%%%%%%%%%%%%%%%%%%%%%%%%%%%%%%%%%%%%%%%%%%%%%%%%%%%%%%%%%%%%%
% Proforma, table of contents and list of figures

\setcounter{page}{1}
\pagenumbering{roman}
\pagestyle{plain}

\chapter*{Proforma}

{\large
\begin{tabular}{ll}
Name:               & \bf John Knox                                 \\
College:            & \bf Churchill College                         \\
Project Title:      & \bf RaceTrace: A real-time group tracking                   \\
                    & \bf app                        \\
Examination:        & \bf Computer Science Tripos, Part II (2013)   \\
Word Count:         & \bf To do\footnotemark[1]                     \\
Project Originator: & Dr John Fawcett                               \\
Supervisor:         & Dr John Fawcett                               \\ 
\end{tabular}
}
\footnotetext[1]{This word count was computed
by {\tt detex diss.tex | tr -cd '0-9A-Za-z $\tt\backslash$n' | wc -w}
}
\stepcounter{footnote}


\section*{Original Aims of the Project}

To create an android application that allows multiple participants of some track based activity to continually track each other's progress. The display would be accurate enough to provide useful location information in such a situation. This translates to having accuracy and latency such that at all times the locations displayed must be either less than ~5 seconds late or less than ~10 metres from the true location.
The communication was to be implemented in more than one way, so that an investigation could, and would be done into the relative effectiveness of at least two different protocols.


\section*{Work Completed}

I have created an Android application that achieves this by first letting users connect to each other in one of two different ways, and then displaying a diagram which shows the relative positions and trails of the other players on the screen, along with that of the user. This diagram is continually updated with the latest known locations of the other users. The inter-device communication has been implemented in two distinct ways, the first using a client-server model, and the second being peer-to-peer with various further protocol options for each model.
The networking technologies used are WiFi and operator provided mobile internet.
I have also conducted a quantitative evaluation of the protocols implemented, ... to do.
The work completed fulfils the project requirements.

\section*{Special Difficulties}

To do.
 
\newpage
\section*{Declaration}

I, John Knox of Churchill College, being a candidate for Part II of the Computer
Science Tripos [or the Diploma in Computer Science], hereby declare
that this dissertation and the work described in it are my own work,
unaided except as may be specified below, and that the dissertation
does not contain material that has already been used to any substantial
extent for a comparable purpose.

\bigskip
\leftline{Signed [signature]}

\medskip
\leftline{Date [date]}

\cleardoublepage

\tableofcontents

%NOT SURE IF THIS IS NEEDED? To do.
%\listoffigures

\newpage
\section*{Acknowledgements}

To do.

%%%%%%%%%%%%%%%%%%%%%%%%%%%%%%%%%%%%%%%%%%%%%%%%%%%%%%%%%%%%%%%%%%%%%%%
% now for the chapters

\cleardoublepage        % just to make sure before the page numbering
                        % is changed

\setcounter{page}{1}
\pagenumbering{arabic}
\pagestyle{headings}

\chapter{Introduction}

\section{Context and Motivation}

With the sudden rise of smart phones in the past decade, new innovations are constantly being discovered. One niche within the applications market is the area of physical sports enhancement. There are now multiple widely used applications on the market offering real-time as well as historic personal tracking and analysis for the duration of some physical activity, be it cycling, running, skiing or something else altogether.
Typically these applications display the location of the user on screen as they advance around some course, and provide statistics along the way such as velocity, distance travelled and many others. They also allow the user to store records of their session, so that future performance can be compared and further analysis done.
However, despite the many features on offer, none of these solutions acknowledge the fact that often the activities involved are group events, thus failing to capitalise on an entire dimension of available data. Namely, the performance of all of the other participants with whom the user may be competing with\footnote{The social networking "share" paradigm has been applied to such applications, allowing others to see user statistics after the event has taken place but, to my knowledge, there is no real-time sharing facility allowing others to compete at the same time}.

In the video game industry it has long been standard practice for racing games to display a simple map of the course on the screen (typically in one of the corners) showing the location of each competitor. This allows the user to continually evaluate their relative position, and stay informed throughout the duration of the race. There are arguments for and against such a feature\footnote{While knowing the location of other participants can increase competitiveness and therefore improve performance, it can also result in degraded performance due to winners competency and losers frustration * economics/psychology reference?}, but its unanimous inclusion in gaming is sufficient proof that it is, at the very least, sometimes desirable.

\section{Initial Direction}

Taking a step back for a moment and considering the alternatives to developing a smart phone application, there are a variety of ways such a system could be implemented. One approach to solving this problem would be to develop an application specific device for the job. If this approach was taken then a multitude of options would open up as to what hardware should be used.
The networking aspect of the device would perhaps be the most interesting with the large variance between available options.
One reasonable option would be to use a private radio system. This would offer many benefits. In a typical use case of the desired system, the distance between participants would not exceed the range of that of available portable license-free radio modules. This means the user could support their own network infrastructure without relying on external providers or being subject to external costs.
While designing a custom device would have some definite advantages, with the widespread adoption of smartphones - most of which have all of the necessary hardware for this task, there is really no need.
On top of that, current top end mobile phones nowadays have capabilities by far exceeding the requirements of such a project, making the design process easier and allowing for more complicated features to be added to the basic idea.
For this reason I have opted to design a smartphone application. At the time of the decision, there were three main mobile application environments, namely Apple's IOS, Android, and Windows Phone 7. The resources offered by each platform do not vary much, so it was a somewhat arbitrary decision of which to choose. The Android platform was chosen out of familiarity with the Java platform as well as having multiple devices at my disposal.

\section{Scope of the Work}

In order to achieve the goals laid out at the start there are three main features that are required. These are outlined below.

\subsection{User Interface}

The app requires a main graphical interface for displaying the position of the other participants in the relation to the user. It was not my intention to produce a production grade user interface but I recognise that this would be a useful addition to the project. The only requirement was that it be effective and reliable, this has been achieved.

\subsection{Session Setup}

In order for inter-device communication, there needs to be a method of synchronisation so that each device knows where to send data to. There are various possible approaches to this problem. I have implemented two alternate methods, each with their own merits.

\subsection{Inter-Device Communication}

The way devices communicate with each other is crucial to the project. I have used the internet API provided by the Android operating system to do all main communication.

\cleardoublepage

\chapter{Preparation}

In this chapter I...

\section{Resources Available}

In this section I shall go through the relevant capabilities of the devices at which I am aiming my application.

\subsection{Networking}

The ability to communicate between devices is vital to this project. Fortunately, there are multiple methods of communication on offer to mobile developers.
Practically all mobile phones are now connected to the internet, many with high speed access thanks to 3G and 4G technologies, as well as having 802.11 WLAN and bluetooth.
Communication can be split into ad-hoc and infrastructure methods.
Bluetooth falls under the Ad-hoc category, where there is no external networking infrastructure, along with WiFi direct, a protocol available to the latest Android devices where WiFi can be used in Ad-hoc mode without the need for an access point.
WiFi is more commonly used in infrastructure mode, as a portal to the wider internet, as well as the mobile internet services such as 3G.
The Android OS also allows devices to share an internet connection via WiFi or Bluetooth.

\subsection{Processor}

The processing power of available Android devices varies widely, from top-end models having quad-core processors and dedicated graphics, to budget devices performing several orders of magnitude worse on CPU benchmarks\footnote{\url{http://www.androidbenchmark.net/cpumark_chart.html}}. Even the low end devices have 

\section{Mobile Networks}

All popular mobile providers provide some wireless internet service, usually accessible anywhere that conventional "phone call coverage" is available. This allows for continuous connection while on the go, and allows applications to interact with servers located anywhere in the world.

\subsection{Reliability}

The portability of mobile phones brings some disadvantages along with it. One being that signal strength varies widely across land, depending on proximity of base stations as well as the characteristics of the surrounding landscape.

Typical uses of sports tracking applications include mountain biking and skiing, activities such as these raise concerns due to the nature of their location, often being within dense areas of forest and in remote areas that wouldn't usually be prioritised for by network providers. For these reasons, signal strength in these locations is typically weak, often cutting out entirely and though significant effort has been made to avoid it\footnote{Refer to that paper about link-layer retransmission to target packet loss} packet loss is still a problem.

This means for an application relying on data coverage for communications would need to have some resilience to such disconnection, and should use the available network capabilities efficiently so that important data is prioritised.

\section{Available Network Technology}

\section{Networking Protocols}

\section{GPS}


\cleardoublepage
\chapter{Implementation}

To do.

\cleardoublepage
\chapter{Evaluation}

To do.

\cleardoublepage
\chapter{Conclusion}

To do.

\cleardoublepage

%%%%%%%%%%%%%%%%%%%%%%%%%%%%%%%%%%%%%%%%%%%%%%%%%%%%%%%%%%%%%%%%%%%%%
% the bibliography

\addcontentsline{toc}{chapter}{Bibliography}
\bibliography{refs}
\cleardoublepage

%%%%%%%%%%%%%%%%%%%%%%%%%%%%%%%%%%%%%%%%%%%%%%%%%%%%%%%%%%%%%%%%%%%%%
% the appendices
\appendix

\chapter{Project Proposal}

%
% Draft #1 (final?)

\vfil

\centerline{\Large Computer Science Project Proposal}
\vspace{0.4in}
\centerline{\Large How to write a dissertation in \LaTeX\ }
\vspace{0.4in}
\centerline{\large M. Richards, St John's College}
\vspace{0.3in}
\centerline{\large Originator: Dr M. Richards}
\vspace{0.3in}
\centerline{\large 14$^{th}$ October 2011}

\vfil


\noindent
{\bf Project Supervisor:} Dr M. Richards
\vspace{0.2in}

\noindent
{\bf Director of Studies:} Dr M. Richards
\vspace{0.2in}
\noindent
 
\noindent
{\bf Project Overseers:} Dr~F.~H.~King  \& Dr~A.~W.~Moore


% Main document

\section*{Introduction, The Problem To Be Addressed}


Many students write their CST dissertations in \LaTeX\ and
spend a fair amount of time learning just how to do that. The purpose of 
this project is to write a demonstration dissertation that explains in
detail how it done.  

This core proposal document will be augmented by a separately-printed
cover sheet at the front and a resource form at the end.  Additional
sheets for risk assessment and human resources may also need to be included.

This document will repeat much of the material that is summarised on the additional sheets.

\section*{Starting Point}

{\em Describe existing state of the art, previous work in this area, libraries and databases to be used.
Describe the state of any existing codebase that is to be built on.  }

I am already able to write prose using the English language. I have an online dictionary. etc..

\section*{Resources Required}

{\em A note of the resources required and confirmation of access.}

For this project I shall mainly use my own quad-core computer that runs Fedora Linux. Backup
will be to github and/or to an SVN repository on an external hard disk that is dumped to writable CD/DVD media.
I have another similar computer to hand should my main machine suddenly fail.
I require no other special resources.

\section*{Work to be done}

{\em Describe the technical work.}

The project breaks down into the following sub-projects:

\begin{enumerate}

\item The construction of a skeleton dissertation with the required 
structure. This involves writing the Makefile and makeing dummy files
for the title page, the proforma, chapters 1 to 5, the appendices and
the proposal.

\item Filling in the details required in the cover page and proforma.

\item Writing the contents of chapters 1 to 5, including examples
of common \LaTeX\ constructs.

\item Adding a example of how to use floating figures and encapsulated
postscript diagrams.

\end{enumerate}

\section*{Success Criterion for the Main Result}


The project will be a success if I have a completed dissertation with the correct chapter
titles and I have achieved my other success criterion, which is to blah ...



\section*{Possible Extensions}

{\em Potential further envisaged evaluation metrics or extensions.}

If I achieve my main result early I shall try the following alternative experiment or method of evaluation ...


\section*{Timetable: Workplan and Milestones to be achieved.}


{\em Perhaps list ten or so  two-week work-packages.}

Planned starting date is 16/10/2011.

\begin{enumerate}

\item {\bf Michaelmas weeks 2-4} Learn to use X. Read book Y. Read papers Z.

\item {\bf Michaelmas weeks 5-6} Do preliminary test of Q.

\item {\bf Michaelmas weeks 7-8} Start implementation of main task A.

\item {\bf Michaelmas vacation} Finish A and start main task B.

\item {\bf Lent weeks 0-2} Write progress report. Generate corpus of test examples. Finish task B.  

\item {\bf Lent weeks 3-5} Run main experiments and achieve working project.

\item {\bf Lent weeks 6-8} Second main deliverable here.

\item {\bf Easter vacation:} Extensions and writing dissertation main chapters.

\item {\bf Easter term 0-2:}  Further evaluation and complete dissertation.

\item {\bf Easter term 3:} Proof reading and then an early submission so as to concentrate on examination revision.

\end{enumerate}


 



\end{document}
