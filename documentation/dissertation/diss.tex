%Useful latex commands from demo diss:
%\noindent followed by a paragraph wont indent the start of it
%{\tt put text here} will put it in monospace(?) font
%\begin{samepage} presumably makes an atomic block
%see demo diss for making latex compile simple block diagrams etc. protocol state machines?
%also see it for adding diagrams. recommends using postscript

\documentclass[12pt,twoside,notitlepage]{report}

\usepackage{a4}
\usepackage{verbatim}
\usepackage{url}
\usepackage{amsmath}

\input{epsf}                            % to allow postscript inclusions
% On thor and CUS read top of file:
%     /opt/TeX/lib/texmf/tex/dvips/epsf.sty
% On CL machines read:
%     /usr/lib/tex/macros/dvips/epsf.tex



\raggedbottom                           % try to avoid widows and orphans
\sloppy
\clubpenalty1000%
\widowpenalty1000%

\addtolength{\oddsidemargin}{6mm}       % adjust margins
\addtolength{\evensidemargin}{-8mm}

\renewcommand{\baselinestretch}{1.1}    % adjust line spacing to make
                                        % more readable

\begin{document}

\bibliographystyle{plain}


%%%%%%%%%%%%%%%%%%%%%%%%%%%%%%%%%%%%%%%%%%%%%%%%%%%%%%%%%%%%%%%%%%%%%%%%
% Title


\pagestyle{empty}

\hfill{\LARGE \bf John Knox}

\vspace*{60mm}
\begin{center}
\Huge
{\bf RaceTrace:\\A real-time group tracking app} \\
\vspace*{5mm}
Computer Science Tripos, Part II \\
\vspace*{5mm}
Churchill College \\
\vspace*{5mm}
\today  % today's date
\end{center}

\cleardoublepage

%%%%%%%%%%%%%%%%%%%%%%%%%%%%%%%%%%%%%%%%%%%%%%%%%%%%%%%%%%%%%%%%%%%%%%%%%%%%%%
% Proforma, table of contents and list of figures

\setcounter{page}{1}
\pagenumbering{roman}
\pagestyle{plain}

\chapter*{Proforma}

{\large
\begin{tabular}{ll}
Name:               & \bf John Knox                                 \\
College:            & \bf Churchill College                         \\
Project Title:      & \bf RaceTrace: A real-time group tracking                   \\
                    & \bf app                        \\
Examination:        & \bf Computer Science Tripos, Part II (2013)   \\
Word Count:         & \bf To do\footnotemark[1]                     \\
Project Originator: & Dr John Fawcett                               \\
Supervisor:         & Dr John Fawcett                               \\ 
\end{tabular}
}
\footnotetext[1]{This word count was computed
by {\tt detex diss.tex | tr -cd '0-9A-Za-z $\tt\backslash$n' | wc -w}
}
\stepcounter{footnote}


\section*{Original Aims of the Project}

To create an android application that allows multiple participants of some track based activity to continually track each other's progress. The display would be accurate enough to provide useful location information in such a situation. This translates to having accuracy and latency such that at all times the locations displayed must be either less than ~5 seconds late or less than ~10 metres from the true location.
The communication was to be implemented in more than one way, so that an investigation could, and would be done into the relative effectiveness of at least two different protocols.


\section*{Work Completed}

I have created an Android application that achieves this by first letting users connect to each other in one of two different ways, and then displaying a diagram which shows the relative positions and trails of the other players on the screen, along with that of the user. This diagram is continually updated with the latest known locations of the other users. The inter-device communication has been implemented in two distinct ways, the first using a client-server model, and the second being peer-to-peer with various further protocol options for each model.
The networking technologies used are WiFi and operator provided mobile internet.
I have also conducted a quantitative evaluation of the protocols implemented, ... to do.
The work completed fulfils the project requirements.

\section*{Special Difficulties}

To do.
 
\newpage
\section*{Declaration}

I, John Knox of Churchill College, being a candidate for Part II of the Computer
Science Tripos [or the Diploma in Computer Science], hereby declare
that this dissertation and the work described in it are my own work,
unaided except as may be specified below, and that the dissertation
does not contain material that has already been used to any substantial
extent for a comparable purpose.

\bigskip
\leftline{Signed [signature]}

\medskip
\leftline{Date [date]}

\cleardoublepage

\tableofcontents

%NOT SURE IF THIS IS NEEDED? To do.
%\listoffigures

\newpage
\section*{Acknowledgements}

To do.

%%%%%%%%%%%%%%%%%%%%%%%%%%%%%%%%%%%%%%%%%%%%%%%%%%%%%%%%%%%%%%%%%%%%%%%
% now for the chapters

\cleardoublepage        % just to make sure before the page numbering
                        % is changed

\setcounter{page}{1}
\pagenumbering{arabic}
\pagestyle{headings}

\chapter{Introduction}

\section{Context and Motivation}

With the sudden rise of smart phones in the past decade, new innovations are constantly being discovered. One niche within the applications market is the area of physical sports enhancement. There are now multiple widely used applications on the market offering real-time as well as historic personal tracking and analysis for the duration of some physical activity, be it cycling, running, skiing or something else altogether.
Typically these applications display the location of the user on screen as they advance around some course, and provide statistics along the way such as velocity, distance travelled and many others. They also allow the user to store records of their session, so that future performance can be compared and further analysis done.
However, despite the many features on offer, none of these solutions acknowledge the fact that often the activities involved are group events, thus failing to capitalise on an entire dimension of available data. Namely, the performance of all of the other participants with whom the user may be competing with\footnote{The social networking "share" paradigm has been applied to such applications, allowing others to see user statistics after the event has taken place but, to my knowledge, there is no real-time sharing facility allowing others to compete at the same time}.

In the video game industry it has long been standard practice for racing games to display a simple map of the course on the screen (typically in one of the corners) showing the location of each competitor. This allows the user to continually evaluate their relative position, and stay informed throughout the duration of the race. There are arguments for and against such a feature\footnote{While knowing the location of other participants can increase competitiveness and therefore improve performance, it can also result in degraded performance due to winners competency and losers frustration * economics/psychology reference?}, but its unanimous inclusion in gaming is sufficient proof that it is, at the very least, sometimes desirable.

\section{Initial Direction}

Taking a step back for a moment and considering the alternatives to developing a smart phone application, there are a variety of ways such a system could be implemented. One approach to solving this problem would be to develop an application specific device for the job. If this approach was taken then a multitude of options would open up as to what hardware should be used.
The networking aspect of the device would perhaps be the most interesting with the large variance between available options.
One reasonable option would be to use a private radio system. This would offer many benefits. In a typical use case of the desired system, the distance between participants would not exceed the range of that of available portable license-free radio modules. This means the user could support their own network infrastructure without relying on external providers or being subject to external costs.
While designing a custom device would have some definite advantages, with the widespread adoption of smartphones - most of which have all of the necessary hardware for this task, there is really no need.
On top of that, current top end mobile phones nowadays have capabilities by far exceeding the requirements of such a project, making the design process easier and allowing for more complicated features to be added to the basic idea.
For this reason I have opted to design a smartphone application. At the time of the decision, there were three main mobile application platforms, namely Apple iOS, Android, and Microsoft Windows Phone 7. The resources offered by each platform do not vary much, so it was a somewhat arbitrary decision of which to choose. The Android platform was chosen out of familiarity with the Java platform as well as having multiple devices at my disposal.

\section{Scope of the Work}

In order to achieve the goals laid out at the start there are three main features that are required. These are outlined below.

\subsection{User Interface}

The app requires a main graphical interface for displaying the position of the other participants in relation to the user. It was not my intention to produce a production grade user interface though I recognise that this would be a useful addition to the project. The only requirement was that it be effective and reliable, this has been achieved.

\subsection{Session Setup}

In order for inter-device communication, there needs to be a method of synchronisation so that each device knows where to send data to. There are various possible approaches to this problem. I have implemented two alternate methods, each with their own merits.

\subsection{Inter-Device Communication}

The way devices communicate with each other is crucial to the project. I have used the internet API provided by the Android operating system to do all main communication.

\cleardoublepage

\chapter{Preparation}

In this chapter I...

\section{Resources Available}

In this section I shall go through the relevant capabilities of the devices at which I am aiming my application.

\subsection{Networking}

The ability to communicate between devices is vital to this project. Fortunately, there are multiple methods of communication on offer to mobile developers.
Practically all mobile phones are now connected to the internet, many with high speed access thanks to 3G and 4G technologies, as well as having 802.11 WLAN and bluetooth.
Communication can be split into ad-hoc and infrastructure methods.
Bluetooth falls under the Ad-hoc category, where there is no external networking infrastructure, along with WiFi direct, a protocol available to the latest Android devices where WiFi can be used in Ad-hoc mode without the need for an access point.
WiFi is more commonly used in infrastructure mode, as a portal to the wider internet, as well as the mobile internet services such as 3G.
The Android OS also allows devices to share an internet connection via WiFi or Bluetooth.

With the android devices available at the time, I had no way to use WiFi direct since this is exclusive to higher versions of Android. The network providers data connection was the chosen method of communications, since it provides uniform access between pairs of devices whenever they have signal, as opposed to some ad-hoc method such as bluetooth where opportunistic transfer would have to be relied on and application performance would vary widely depending on the differing distributions of participants. For example chains of players where each is within range of the next would allow for a fully connected network, whereas any other arrangement would not, resulting in partial data showing on screen until the networks reconnect. In the worst case if no players were within ad-hoc range of any other then no data would be transferred rendering the application useless.

Adding ad-hoc communication as an additional feature to the system I have built would be advantageous, and will be explored more in future work section\footnote{Resilience, importance of nearest neighbours.}, but developing an opportunistic network system would considerably increase the scope of the project so it has not been explored further.

\subsection{Processor}

The processing power of available Android devices varies widely, from top-end models having quad-core processors and dedicated graphics, to budget devices performing several orders of magnitude worse on CPU benchmarks\footnote{\url{http://www.androidbenchmark.net/cpumark_chart.html}}. Even the low end devices are sufficient for the computation required for this task. The requirements are:
\begin{itemize}
\item{Send and receive multiple network packets per second}
\item{Encrypt and decrypt multiple packets per second using some encryption scheme}
\item{Scan through arrays spanning hundreds of data points, drawing each to the screen without delaying the user}
\end{itemize}

\subsection{Memory}

In order to function properly, each device must record a trace of its history and that of the other devices. This has the potential to be a large amount of data so the requirement was estimated before proceeding with the project.
A typical use scenario would not normally exceed 4 hours.
At one update per second per device the required memory to store the history would be:
\begin{equation}
size = 1\times60\times60\times4\times12B
\end{equation}
\begin{equation*}
= 172,800B
\end{equation*}
That is, 169KB for each device. Clearly the requirement will scale linearly with the number of devices.

All Android devices provide each application with at least 16MB of allocatable memory. Bearing this in mind, the limit on number of users will lie somewhere between 50 and 100, far more than the number in a typical use case.

\section{Networking}

All popular mobile providers provide some wireless internet service, usually accessible anywhere that conventional "phone call coverage" is available. This allows for continuous connection while on the go, and allows applications to interact with servers located anywhere in the world.

\subsection{Reliability of Mobile Networks}

The portability of mobile phones brings some disadvantages along with it. One being that signal strength varies widely across land, depending on proximity of base stations as well as the characteristics of the surrounding landscape.

Typical uses of sports tracking applications include mountain biking and skiing, activities such as these raise concerns due to the nature of their location, often being within dense areas of forest and in remote areas that wouldn't usually be prioritised for by network providers. For these reasons, signal strength in these locations is typically weak, often cutting out entirely and though significant effort has been made to avoid it\footnote{Refer to that paper about link-layer retransmission to target packet loss} packet loss is still a problem.

This means an application relying on data coverage for communications would need to have some resilience to disconnection, and should use the available network throughput efficiently so that important data is prioritised.

\subsection{Network Protocols}

\subsubsection{Client-Server vs Peer-to-Peer Systems}
Given that the phone data connection is to be used as the communications "port", there are two ways the system could be achieved.

One approach uses the support of servers in the cloud to serve client systems and act as intermediaries. The other approach is to use only client systems and no external processing other than the network infrastructure.

The main advantage to the peer-to-peer approach is scalability. If the application was to be widely used then a peer-to-peer system would be self sufficient and allow unbridled scale.

The client server system on the other hand would require servers to be maintained either at the cost of the developer or the user. If the application package provided this service then more resources would have to be provided as the number of users increases. As well as this it would be a single point of failure whereas peer-to-peer systems are only reliant on themselves.

There are advantages to the client server model however, for one it avoids the NAT traversal problem outlined below. In the event of high packet loss, considered likely given the application, packets can be lost when transmitted from the phone to the base station and vice-versa. For peer-to-peer and client server systems alike, the chance of a successful transmission with one attempt at sending is:
\begin{equation}
P_{success} = P \times P
\end{equation}
However when repeat attempts are allowed from the intermediate server, the probability of success lies in favour of the client server system:
\begin{equation}
P_{success} = P \times (P + \gamma)
\end{equation}
where
\begin{equation}
\gamma = (1-P)P + (1-P)^2P + (1-P)^3P + ...
\end{equation}
Hence even when repeat transmissions are enabled from the initial sender (in this case the Android device), the client and server model will always have a probabilistic advantage where there is packet loss\footnote{Assuming that packet loss between the base station and server is negligible}.

\subsection{Network Address Translation}

With most internet services still using IPv4, addresses now come at a premium with many ISPs using Network Address Translation (NAT) to expand their internal network address space. With the transient nature of mobile phone data connections, most if not all mobile ISPs fall into this category. One problem with this technology is that it can make peer-to-peer connections difficult or even impossible.

NAT occurs at the router between an internal network and some wider network. This router allocates addresses to all of the nodes in the internal network, but has it's own external network address. When an internal node sends a packet to some node in the external network, the router translates the ...

Tests were conducted to find out whether peer-to-peer communication would be possible over the mobile data networks.
Using UDP hole punching I was able to connect to an Android device over the 3G network provided by Three. Ordinarily this would provide the means to achieve peer-to-peer communication, by using the same port number on two devices, each punching a hole towards the other to open up a channel. In the case of Three, there are further complications meaning that packets can successfully be sent from external networks, such as the Internet connection provided by Cambridge University, but sending packets from within the Three network to other addresses in the Three network could not be done. From this I concluded that the network provider doesn't allow intra-network communication\footnote{speculate block malliscoius apps} meaning true peer-to-peer protocols are not possible when limited to this network provider. Since external connections were successfully achieved, I assume that peer-to-peer communications would be possible when a mix of network providers is used. When testing, I had access to two 3G data networks, Three and Vodafone. Unlike Three, in my experiments Vodafone always remapped the port numbers for outgoing messages, meaning the UDP hole punching method of NAT traversal would not work, disallowing peer-to-peer traffic between Three and Vodafone.
In light of these discoveries, it was clear that a peer-to-peer system would not be universally accessible, but would work in theory, for example between two networks operating using the system that Three uses. The transition from IPv4 to IPv6 should eliminate the need for NAT, putting all devices in the same address space, making peer-to-peer mobile applications more practical and universally accessible.

For the above reasons, doing an implementation of the system using a peer-to-peer protocols would still have benefits, so I continued with the plan to make both a client server implementation and a peer-to-peer one.

\section{Location Data}

\subsection{Obtaining a Location}

Most Android phones are able to use both GPS satellites and the mobile phone network to triangulate their location. If connected to Wi-Fi this can also be used to fetch location data.
The android OS can provide location estimates from all of the above sources, when available, along with an estimate of the accuracy. This allows applications to combine the different location providers and use the location data of the one that is most accurate at the time. This provides some redundancy meaning long intervals without a location update are rare. The power consumption of the device increases as more location providers are used, with GPS being the most power hungry. For this reason, update intervals can be specified so that the providers will only update their location value at specific time intervals. Clearly the interval size depends on the application, with this particular application, aiming for real time updates and bearing in mind the competitive purpose of the application, relatively frequent updates are desired.
In all tests conducted, the frequency of updates used was 1 second though this value is contained in the configuration file and can easily be altered in the application settings. It was observed that on the test devices used, battery consumption did not suffer severely even with this frequent update period. [maybe this should be in implementation] + about using combination of porviders with network providing initial while GPS locates it.

\subsection{Displaying Locations}
When drawing the trail of a user to the screen, it is important for the image to appear in agreement with the path the user has experienced. Since we, as humans, don't notice the curvature of the earth in everyday activities, we perceive it as flat. This means when we walk the perimeter of a "square" on the surface of the earth, we expect to see the drawn path appear as a square.

However the data provided by each of the location providers comes in the form of latitude, longitude, altitude, speed, and time values. Because the shape of the earth is closer to that of an ellipsoid than a sphere, simply plotting the latitude and longitude as y and x coordinates on the screen would cause the image to appear skewed. To account for the uneven form of the planet, we must apply a coordinate transform to map the data onto a geographic coordinate system. There are four main coordinate systems in use, latitude and longitude (LL), Universal Transverse Mercator (UTM), Universal Polar Stereographic (UPS), and Ordinance Survey Great Britain (OSGB). UPS is a system designed to cover the polar regions, which are not covered by UTM. I discarded the option of proceeding with UPS because for the purposes of this project, operating in the polar regions is not my concern, though this decision may have to be revisited if further development takes place. OSGB covers the area in and around Great Britain, so this immediately makes the system slightly less desirable, but again for the purposes of this project, it would suffice, leaving UTM and OSGB remaining.

Both transforms are based on an ellipsoidal model of the earth, though UTM uses the more universal WGS84 model. Being a transverse Mercator projection, UTM has the property of being conformal, meaning that it preserves angles but distorts distance and area\footnote{\url{http://en.wikipedia.org/wiki/Universal_Transverse_Mercator_coordinate_system}}, whereas OSGB is a Mercator projection meaning it is not conformal and hence preserves distance and area, and distorts angles.

In order to tranform coordinates from lat/long to one of these systems, there is not just a single map transformation to apply. For example, in UTM the surface of the planet is divided into 60 geographical zones based on longitude and a different transverse mercator projection is used for each zone. This means calculation isn't trivial, but fortunately there are various open source libraries available that offer conversion between the systems. Of these I chose JCoord because of those available in Java with suitable licensing, it was the most lightweight and provides both OSGB and UTM coordinate systems.

To decide which projection, if any would be needed, I used the location plotting function of the application (which was developed at the beginning of the project and is explained in ...) to trace my path when walking in certain shapes.
Walking along the perimeter of an "ultimate"\footnote{Ultimate (more commonly known as ultimate frisbee is a sport played on a rectangular pitch with rectangular end zones at each side} pitch I was able to display the shape when plotted in the different coordinate systems to compare the shape and detect any severe distance scaling using the relative sizes of the end zones as well as any angle distortion from the well defined right angled corners of the pitch. Using a sports pitch also has the advantage that the surface is level and flat.
When drawing the raw longitude and latitude as x and y coordinates, there was clear distortion in the angles of the trace displayed, causing the pitch to resemble a parallelogram rather than a rectangle. This deformity was also observed, to a slightly lesser extend when using the OSGB transform. However, when the UTM transform was used, there was a large improvement in the shape of the trace displayed. Upon careful inspection of the UTM trace the edges were found to be not perfectly perpendicular, but this error was not easily noticeable and certainly good enough for the desired application. There were also no observed differences between the areas of the end zones.

With these results, I chose to use the UTM coordinate system as it displayed a much more intuitive representation of the path in the location where it was tested\footnote{Churchill College Grounds, Cambridge} than the other transforms.

\cleardoublepage
\chapter{Implementation}

To do.

\cleardoublepage
\chapter{Evaluation}

To do.

Include:
\begin{itemize}
\item{Differing characteristics of UK phone networks (Vodafone and Three)}
\end{itemize}

\cleardoublepage
\chapter{Conclusion}

To do.

\cleardoublepage

%%%%%%%%%%%%%%%%%%%%%%%%%%%%%%%%%%%%%%%%%%%%%%%%%%%%%%%%%%%%%%%%%%%%%
% the bibliography

\addcontentsline{toc}{chapter}{Bibliography}
\bibliography{refs}
\cleardoublepage

%%%%%%%%%%%%%%%%%%%%%%%%%%%%%%%%%%%%%%%%%%%%%%%%%%%%%%%%%%%%%%%%%%%%%
% the appendices
\appendix

\chapter{Project Proposal}

%
% Draft #1 (final?)

\vfil

\centerline{\Large Computer Science Project Proposal}
\vspace{0.4in}
\centerline{\Large How to write a dissertation in \LaTeX\ }
\vspace{0.4in}
\centerline{\large M. Richards, St John's College}
\vspace{0.3in}
\centerline{\large Originator: Dr M. Richards}
\vspace{0.3in}
\centerline{\large 14$^{th}$ October 2011}

\vfil


\noindent
{\bf Project Supervisor:} Dr M. Richards
\vspace{0.2in}

\noindent
{\bf Director of Studies:} Dr M. Richards
\vspace{0.2in}
\noindent
 
\noindent
{\bf Project Overseers:} Dr~F.~H.~King  \& Dr~A.~W.~Moore


% Main document

\section*{Introduction, The Problem To Be Addressed}


Many students write their CST dissertations in \LaTeX\ and
spend a fair amount of time learning just how to do that. The purpose of 
this project is to write a demonstration dissertation that explains in
detail how it done.  

This core proposal document will be augmented by a separately-printed
cover sheet at the front and a resource form at the end.  Additional
sheets for risk assessment and human resources may also need to be included.

This document will repeat much of the material that is summarised on the additional sheets.

\section*{Starting Point}

{\em Describe existing state of the art, previous work in this area, libraries and databases to be used.
Describe the state of any existing codebase that is to be built on.  }

I am already able to write prose using the English language. I have an online dictionary. etc..

\section*{Resources Required}

{\em A note of the resources required and confirmation of access.}

For this project I shall mainly use my own quad-core computer that runs Fedora Linux. Backup
will be to github and/or to an SVN repository on an external hard disk that is dumped to writable CD/DVD media.
I have another similar computer to hand should my main machine suddenly fail.
I require no other special resources.

\section*{Work to be done}

{\em Describe the technical work.}

The project breaks down into the following sub-projects:

\begin{enumerate}

\item The construction of a skeleton dissertation with the required 
structure. This involves writing the Makefile and makeing dummy files
for the title page, the proforma, chapters 1 to 5, the appendices and
the proposal.

\item Filling in the details required in the cover page and proforma.

\item Writing the contents of chapters 1 to 5, including examples
of common \LaTeX\ constructs.

\item Adding a example of how to use floating figures and encapsulated
postscript diagrams.

\end{enumerate}

\section*{Success Criterion for the Main Result}


The project will be a success if I have a completed dissertation with the correct chapter
titles and I have achieved my other success criterion, which is to blah ...



\section*{Possible Extensions}

{\em Potential further envisaged evaluation metrics or extensions.}

If I achieve my main result early I shall try the following alternative experiment or method of evaluation ...


\section*{Timetable: Workplan and Milestones to be achieved.}


{\em Perhaps list ten or so  two-week work-packages.}

Planned starting date is 16/10/2011.

\begin{enumerate}

\item {\bf Michaelmas weeks 2-4} Learn to use X. Read book Y. Read papers Z.

\item {\bf Michaelmas weeks 5-6} Do preliminary test of Q.

\item {\bf Michaelmas weeks 7-8} Start implementation of main task A.

\item {\bf Michaelmas vacation} Finish A and start main task B.

\item {\bf Lent weeks 0-2} Write progress report. Generate corpus of test examples. Finish task B.  

\item {\bf Lent weeks 3-5} Run main experiments and achieve working project.

\item {\bf Lent weeks 6-8} Second main deliverable here.

\item {\bf Easter vacation:} Extensions and writing dissertation main chapters.

\item {\bf Easter term 0-2:}  Further evaluation and complete dissertation.

\item {\bf Easter term 3:} Proof reading and then an early submission so as to concentrate on examination revision.

\end{enumerate}


 



\end{document}
